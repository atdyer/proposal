\documentclass[11pt,conference]{IEEEtran}

\usepackage{cite}
\usepackage{url}

\def\BibTeX{{\rm B\kern-.05em{\sc i\kern-.025em b}\kern-.08em
    T\kern-.1667em\lower.7ex\hbox{E}\kern-.125emX}}

\begin{document}

\title{Sparse}
\author{
  \IEEEauthorblockN{Tristan Dyer}
  \and
  \IEEEauthorblockN{John Baugh}
}

\maketitle

\section{Introduction}

%Computations on sparse matrices are prevalent in scientific and engineering software.  
Sparse matrix data formats are able to compress large matrices with a small number of non-zero elements into a more efficient representation.  
%The development of software that makes use of sparse matrices is a tedious and often error-prone task because of the myriad data formats and complexities involved in tuning the operations on these formats to achieve an efficient implementation.
Scientific and engineering software that make use of sparse matrix formats are often implemented in low-level imperative languages such as C++ and Fortran.  The optimized nature of these software often means that the structural organization of sparse matrix formats and mathematical computations are heavily intertwined.  Additionally, the myriad data formats and complexities involved in tuning the operations on these formats to achieve efficient implementations means that the development of software that makes use of sparse matrices is a tedious and often error-prone task.

A number of approaches have been taken in order to address these issues.  Object-oriented libraries such as PETSc~\cite{petsc2019} and Eigen~\cite{eigenweb2010} provide data abstractions targeted towards specific classes of solvers.  These libraries provide templates that allow developers to assemble sparse matrices without having to address the structural complexities that a specific format may present.  These matrices can then be used in a variety of solvers, given that the format is supported.

Alternatively, there is a body of research that takes the approach of designing and building compilers capable of automatically making decisions about sparse matrix formats and operations.  Some compilers~\cite{bik1995, bik1996} allow developers to work with dense matrices in code, generating a sparse matrix program at compile time.  Others~\cite{kotlyar1997} allow the user to provide the compiler with an abstract description of a sparse format, from which the compiler can make automatic optimizations in code that accesses the sparse data.

We describe an approach that allows developers to reason about the inherent complexities of sparse matrix formats and operations and to determine invariants that can be used to verify implementations.
%Rather than hiding the complexities of sparse matrix formats by providing data abstractions in code or automatic optimizations in compilers, we propose an approach that allows developers to reason about these complexities.  
Elements of this approach include declarative modeling and automatic, push-button analysis using the Alloy Analyzer~\cite{jackson2012}, a lightweight bounded model checking tool.  Characteristics of sparse matrices, with their numerous representations and ...hard-to-get-right-implementation-details... are approached using abstraction based methods~\cite{clarke1994}, including data abstraction~\cite{dingel1995} and predicate abstraction~\cite{graf1997}, data and functional refinement~\cite{woodcock1996}, and other techniques, manually, as part of a modeling process...

The benefits of this approach lie in its generality.  By using Alloy to perform the modeling and analysis, the modeler may choose the programming language that best fits their needs when transitioning from model to implementation.
Can reason about existing software.
Can reason about design of new sparse matrix libraries.
Can reason about compiler design.

\section{Lightweight Formal Methods}

...
An additional aspect of lightweight formal methods is an incremental style of modeling, which tools like Alloy support by offering immediate feedback while models are being constructed: we start with a minimal set of constraints and ``grow'' them via conjunction.

\subsection{Alloy}

The tool used in our approach is Alloy, a declarative modeling language combining first-order logic with relational calculus and associated quantifiers and operators, along with transitive closure.  It offers rich data modeling features based on class-like structures and an automatic form of analysis that is performed within a bounded scope using a SAT solver.  For \emph{simulation}, the analyzer can be directed to look for instances satisfying a property of interest.  For \emph{checking}, it looks for an instance violating an assertion: a counterexample.  The approach is \emph{scope complete} in the sense that all cases are checked within user-specified bounds.  Alloy's logic supports three distinct styles of expression, that of predicate calculus, navigation expressions, and relational calculus.  The language used for modeling is also used for specifying properties of interest and assertions.  Alloy supports expressions with integer values and basic arithmetic operations.

\subsection{Data Abstraction}

Proof obligations: mathematical formula to be proven in order to ensure that a component is correct.

Start example here.

\subsection{Data Refinement}

The process of data refinement involves removing nondeterminism, or uncertainty, from an abstract model~\cite{woodcock1996}.  While an abstract model may omit certain design choices, a refinement can resolve some of these choices, removing uncertainty and approaching the level of a concrete implementation.

Diagram of refinement here.

Abstraction function and representation invariant discussion.

Continue example with refinement here.

\section{Sparse Matrix Models}

In this section we describe four models.  First we describe an abstract model of a matrix that does not represent any specific data format, leaving out all implementation details and notions of sparsity. Then we describe three refinements of this model that introduce structure that supports a sparse representation: the DOK format, the ELL format, and the CSR format.  In each of these refinements we determine the appropriate representation invariants as well as the abstraction functions and show that the refinement is valid.

\subsection{Abstract Sparse Matrix}

\begin{figure}
\centering
Value Model.
\caption{Abstract Model of Values}
\label{mod:value}
\end{figure}

By design, Alloy provides no means of working with floating point values.  It includes integers, but in a limited scope, and so they are not useful when attempting to work with the complete integer set, as might be the case when considering values that could be stored in a sparse matrix.  This is inconsequential, however, as we only aim to reason about the structural complexities of sparse matrix formats.  Therefore, it is sufficient to create an abstract distinction between zero and non-zero values.  The model in \figurename~\ref{mod:value} contains a Value signature, representing any numerical value.  The Zero signature, an extension of Value, represents the value zero.  Depending on the scope, this creates an abstract set of ``values'' that we can use to populate matrices in our models.

\begin{displaymath}
Value = \{Zero, Value_0, Value_1, \ldots, Value_n\}
\end{displaymath}

%An abstract model of a matrix is given in \figurename~\ref{mod:abstract}.

%\begin{figure}
%\centering
%Abstract Model.
%\caption{Abstract Matrix Model.}
%\label{mod:abstract}
%\end{figure}

The representation invariant:

\begin{itemize}
	\item Number of rows and columns greather than or equal to zero
	\item All index values fall within bounds
	\item Total number of values in the matrix is \(nrows \times ncols\)
	\item Every \(i, j\) index pair appears in the matrix
\end{itemize}

\subsection{DOK Format}

The dictionary of keys (DOK) format makes use of dictionaries, or associative arrays, to store key, value pairs.  Sparse matrices are stored such that pairs of row, column indices are used as keys to reference stored values.  The DOK format is frequently used~\cite{scipy, eigenweb2010} in the assembly of sparse matrices because of the efficient \(\mathcal{O}(1)\) access to individual elements provided by the associative array format.  However, because associative arrays make no guarantees about memory locality, iterating over all values in the matrix can be inefficient.  As a result, it is common to use this format to incrementally assemble the full matrix and convert it to another sparse matrix format that allows for more efficient operations on matrices.

Why model the DOK format?  It's frequently used.  The model is fairly simple and gives us an idea for the nature of the abstraction functions. 

The abstraction function:

\begin{itemize}
  \item \(nrows\) equal to number of rows in matrix
  \item \(ncols\) equal to number of columns in matrix
  \item All \((i, j), v\) key, value pairs map to corresponding values in the matrix
  \item All \((i, j)\) pairs \emph{not} in DOK map to zeros in the matrix
\end{itemize}

The representation invariant:

\begin{itemize}
  \item The integer value \(nrows \geq 0\)
  \item The integer value \(ncols \geq 0\)
	\item There are no zeros stored as values
	\item All row, column pairs fall within bounds
	\item All row, column pairs may appear only once
\end{itemize}

\subsection{ELL Format}

ELL.

\subsection{CSR Format}

CSR.

Need to describe the ``get'' predicate and why it is needed.

The abstraction function:

\begin{itemize}
  \item \(nrows\) equal to number of rows in matrix
  \item \(ncols\) equal to number of columns in matrix
  \item For all integers \(i \in [0, nrows)\) and \(j \in [0, ncols)\):
  \begin{itemize}
    \item The value of the ``get'' predicate at \((i, j)\) is equal to the value at location \((i, j)\) in the matrix.
  \end{itemize}
\end{itemize}

The representation invariant:

\begin{itemize}
  \item The integer value \(nrows \geq 0\)
  \item The integer value \(ncols \geq 0\)
	\item There are no zeros stored as values
	\item All values in IA \(\leq nrows \times ncols\)
	\item All values in JA \(< ncols\)
	\item The first value in IA is zero
	\item The last value in IA is the length of A
	\item The last value of IA is not repeated
	\item A and JA are the same length
	\item The max length of A is \(nrows \times ncols\)
	\item The max length of IA is \(nrows + 1\)
	\item The values in IA are monotonically increasing
	\item The difference between any two consecutive values in IA must be \(\leq ncols\)
	\item Values in JA must be unique per row
\end{itemize}

\bibliographystyle{IEEEtran}
\bibliography{IEEEabrv,sparse,mybib}

\end{document}
