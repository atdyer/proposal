\documentclass[../../proposal.tex]{subfiles}

\begin{document}

Called a third pillar science, computation is an indispensable tool
not only for scientists, but for engineers who simulate physical and
natural processes to evaluate design alternatives.  Recent studies on
reliability, reproducibility of results, and productivity have cast
concern over what many have suspected or experienced firsthand, that
existing practices of constructing scientific software are inadequate
and limiting the pace of technological advancement.  A disconnect
between modern software engineering practice and scientific
computation is apparent, and yet the unique challenges facing
developers of scientific software must also be recognized: the lack of
test oracles, software lifetimes and evolving needs that span decades,
and the competing objectives of performance, maintainability, and
portability.

I seek to address fundamental design and quality assurance challenges
that are intrinsic to scientific computation and related types of
numerical software.  While numerous directions might be taken, my
premise and motivating viewpoint is the central role that modeling can
and must play in the process of designing and working with complex
artifacts, including scientific programs.  Culturally, the fit may be
a natural one: scientists and engineers are accustomed to working with
models anyway, and with the kind of automatic, push-button analysis
supported by some state-based formalisms, those who develop software
can focus on modeling and design instead of theorem proving.

\end{document}